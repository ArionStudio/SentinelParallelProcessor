
% !TeX program = pdflatex
\documentclass[12pt,a4paper]{article}

% Essential packages
\usepackage[utf8]{inputenc}
\usepackage[T1]{fontenc}
\usepackage[polish]{babel}  % For Polish language support
\usepackage{amsmath}
% \usepackage{amsfonts} % Not strictly needed for this simplified version
% \usepackage{amssymb} % Not strictly needed for this simplified version
% \usepackage{graphicx} % Not strictly needed for this simplified version
\usepackage{hyperref}
% \usepackage{listings} % Not strictly needed for this simplified version
\usepackage{xcolor}
\usepackage{geometry}
% \usepackage{enumitem} % Not strictly needed for this simplified version
\usepackage{booktabs} % For better tables
\usepackage{float}    % For table placement
\usepackage{array}    % For more complex table column definitions

% Document settings
\geometry{
    a4paper,
    margin=2.5cm
}

% Hyperref settings
\hypersetup{
    colorlinks=true,
    linkcolor=blue,
    filecolor=magenta,
    urlcolor=cyan,
    pdftitle={Specyfikacja Projektu: Równoległy System Analizy Roślinności Sentinel-2},
    pdfauthor={Adrian Rybaczuk, Bartosz Cylwik},
    pdfsubject={Przetwarzanie Rozproszone i Równoległe},
    pdfkeywords={Sentinel-2, NDVI, NDMI, Przetwarzanie Równoległe, Python, GUI}
}

% Title information
\title{Specyfikacja Projektu: Równoległy System Analizy Roślinności Sentinel-2}
\author{Adrian Rybaczuk 318483, Bartosz Cylwik 325457}
\date{\today}

\begin{document}
\selectlanguage{polish} % Set language to Polish
\maketitle

\begin{abstract}
\noindent Niniejszy dokument przedstawia specyfikację projektu systemu desktopowego do analizy roślinności z wykorzystaniem danych satelitarnych Sentinel-2. Głównym celem jest implementacja mechanizmu obliczania wskaźników NDVI i NDMI z zastosowaniem technik przetwarzania równoległego w celu optymalizacji wydajności. Projekt obejmuje również stworzenie interfejsu graficznego użytkownika (GUI) do interakcji z systemem.
\end{abstract}


\section*{Wybrane Technologie}
\label{sec:technologies}
Poniższa tabela przedstawia wybrane technologie wraz z ich głównym zastosowaniem w projekcie.

\begin{table}[H]
    \centering
    \caption{Wybrane technologie i ich zastosowanie.}
    \label{tab:technologies}
    \begin{tabular}{>{\raggedright\arraybackslash}p{0.4\textwidth} >{\raggedright\arraybackslash}p{0.5\textwidth}}
        \toprule
        \textbf{Technologia} & \textbf{Zastosowanie w Projekcie} \\
        \midrule
        Język Programowania: Python & Główny język implementacji logiki aplikacji, obliczeń i GUI. \\
        \addlinespace
        Przetwarzanie Równoległe: \texttt{multiprocessing} (Python) & Równoległe wykonywanie obliczeń indeksów NDVI/NDMI na wielu rdzeniach CPU. \\
        \addlinespace
        Przetwarzanie Danych Geoprzestrzennych: Rasterio (z GDAL) & Odczyt, zapis i podstawowe operacje na danych rastrowych Sentinel-2 (format GeoTIFF). \\
        \addlinespace
        Obliczenia Numeryczne: NumPy & Wydajne operacje na tablicach (pikselach obrazów) podczas obliczania indeksów. \\
        \addlinespace
        Interfejs Graficzny Użytkownika (GUI): PyQt6 (lub Tkinter) & Tworzenie interaktywnego interfejsu dla użytkownika (wczytywanie danych, wybór AOI, wizualizacja). \\
        \addlinespace
        Wizualizacja Danych: Matplotlib & Wyświetlanie przetworzonych map NDVI/NDMI w interfejsie graficznym. \\
        \addlinespace
        Format Danych Wyjściowych: GeoTIFF & Standardowy format zapisu przetworzonych map geoprzestrzennych. \\
        \bottomrule
    \end{tabular}
\end{table}

\end{document}
