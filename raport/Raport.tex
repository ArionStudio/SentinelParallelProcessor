% !TeX program = pdflatex
\documentclass[12pt,a4paper]{article}

% Essential packages
\usepackage[utf8]{inputenc}
\usepackage[T1]{fontenc}
\usepackage[polish]{babel}
\usepackage{csquotes}
\usepackage{amsmath}
\usepackage{hyperref}
\usepackage{xcolor}
\usepackage{geometry}
\usepackage{booktabs}
\usepackage{float}  
\usepackage{array}    
\usepackage[backend=biber,style=numeric,sorting=none]{biblatex}
\usepackage{graphicx}
\usepackage{tocloft}  % For customizing table of contents
\usepackage{enumitem} % For better list formatting

\geometry{
    a4paper,
    margin=2.5cm
}

% Table of contents settings
\renewcommand{\cftsecleader}{\cftdotfill{\cftdotsep}}
\renewcommand{\cftsubsecleader}{\cftdotfill{\cftdotsep}}
\setcounter{tocdepth}{2}  % Show sections and subsections in TOC
\setcounter{secnumdepth}{2}  % Number sections and subsections

% Hyperref settings
\hypersetup{
    colorlinks=true,
    linkcolor=blue,
    filecolor=magenta,
    urlcolor=cyan,
    pdftitle={Raport końcowy: Równoległy System Analizy Roślinności Sentinel-2},
    pdfauthor={Adrian Rybaczuk, Bartosz Cylwik},
    pdfsubject={Przetwarzanie Rozproszone i Równoległe},
    pdfkeywords={Sentinel-2, NDVI, NDMI, Przetwarzanie Równoległe, Python, GUI}
}


\title{Projekt zaliczeniowy\\\large Raport końcowy}
\author{Zespół Projektowy nr 1 \\
    \begin{tabular}{ll}
        \textbf{Adrian Rybaczuk} & \textbf{318483} \\
        \textbf{Bartek Cylwik} & \textbf{325457} \\
    \end{tabular}
}
\date{\today}


% Bibliography file
\addbibresource{references.bib}
\selectlanguage{polish} % Set language to Polish
\begin{document}

\maketitle

\begin{abstract}
  Niniejszy raport przedstawia podsumowanie projektu grupowego, systemu do równoległego przetwarzania obrazów satelitarnych Sentinel-2, umożliwiającego efektywną analizę zmian roślinności z wykorzystaniem wskaźników NDVI i NDMI oraz przyjaznego interfejsu graficznego.
  \end{abstract}
  

\tableofcontents
\newpage


\section{Wprowadzenie}
Celem projektu była implementacja (wraz z GUI) systemu równoległego / rozproszonego mechanizmu wyliczania indeksów wilgotności (NDMI) i wegetacji (NDVI) roślin na zobrazowaniach satelitarnych Sentinel-2 do stwierdzenia występowania lub zaniku roślinności na wskazanym obszarze.
Na podstawie tak zdefiniowanego zadania mogliśmy wyznaczyć poszczególne elementy projektu do realizacji:
\begin{itemize}
    \item Uzyskanie wskaźników NDVI i NDMI
    \begin{itemize}
        \item Research na temat wskaźników NDVI i NDMI
        \item Pobranie niezbędnych wskaźników z zobrazowania satelitarnego Sentinel-2
        \item Zapisanie danych pobranych z API jako "cache" w celu uniknięcia ponownego pobierania danych
        \item Identyfikacja potęcjalnych problemów związanych z przetwarzaniem danych
        \item Przeprowadzenie obliczeń wskaźników
    \end{itemize}
    \item Interfejs graficzny
    \begin{itemize}
        \item Ekran uwierzytelniania z API Sentinel-Hub
        \item Ekran pozwalający na wybór zakresu czasowego oraz obszaru analizy
        \item Ekran z wynikami analizy w postaci mapy z zaznaczonymi obszarami roślinności
    \end{itemize}
    \item Równoległe przetwarzanie
    \begin{itemize}
        \item Wybranie obszarów programu wymagających przetworzenia równoległego / rozproszonego
        \item Wybranie pomiedzy przetwarzaniem na CPU a GPU
        \item Implementacja mechanizmu równoległego / rozproszonego przetwarzania
    \end{itemize}
    \item Wybranie technologi do realizacji projektu
\end{itemize}

\newpage

\section{Kroki podjęte na początku pracy}

\subsection{Research wskaźników NDVI i NDMI}

Aby uzyskać wskaźniki NDVI i NDMI, rozpoczęliśmy od analizy literatury i dokumentacji dotyczącej tych indeksów, przechodząc od ogólnych informacji do szczegółowych aspektów ich wyznaczania.

\textbf{NDVI} (Normalized Difference Vegetation Index) \cite{ndvi_docs} jest prostym wskaźnikiem ilościowym służącym do klasyfikacji wegetacji roślin. Jego wartość mieści się w przedziale od $-1$ do $1$:
\begin{itemize}
    \item Wartości ujemne (zbliżone do $0$) wskazują na obecność wody.
    \item Zakres $-0.1$ do $0.1$ odpowiada obszarom jałowym (skały, piasek, śnieg).
    \item Przedział $0.2$ do $0.4$ oznacza niską roślinność (zarośla, łąki).
    \item Wysokie wartości (bliskie $1$) wskazują na bujną roślinność (np. lasy deszczowe).
\end{itemize}

Dzięki temu NDVI jest dobrym wskaźnikiem obecności i kondycji roślinności na danym obszarze.

Wskaźnik NDVI definiuje się wzorem:
\[
\mathrm{NDVI} = \mathrm{Index}(\mathrm{NIR}, \mathrm{RED}) = \frac{\mathrm{NIR} - \mathrm{RED}}{\mathrm{NIR} + \mathrm{RED}}
\]

W przypadku danych Sentinel-2, indeks ten wyznaczamy na podstawie kanałów B8 (NIR) i B4 (RED):
\[
\mathrm{NDVI} = \mathrm{Index}(\mathrm{B8}, \mathrm{B4}) = \frac{\mathrm{B8} - \mathrm{B4}}{\mathrm{B8} + \mathrm{B4}}
\]

\textbf{NDMI} (Normalized Difference Moisture Index) \cite{ndmi_docs} jest znormalizowanym wskaźnikiem wilgotności który do wyznaczenia wilgotności wykorzystuje pasma NIR i SWIR.
\begin{itemize}
    \item Wartosci ujemne (zblizone do -1) wskazuja na bardzo suche obszary (brak roslnosci, pustynia, obszary zabudowane)
    \item Wartosci od -0.2 do 0.2 wskazuja na glebe wysychającą (rzadka trawa, trawy wysychające, krzewy w stanie stresu wodnego)
    \item Wartosci ponizej do 0.2 do 0.4 odpowiadaja umiarkowanemu nawodnieniu (trawy, pastwiska)
    \item Wartosci od 0.4  Wskazują na obszary o zdrowy stan nawodnienia (uprawy w pełni sezonu wegetacyjnego, lasy liściaste w strefie umiarkowanej) 
    \item Wartosci powyzej 0.4 wskazuja na obszary o wysokiej wilgotnosci (lasy deszczowe, obszary podmokłe)
\end{itemize}

Wskaźnik NDMI definiujemy wzorem:
\[
\mathrm{NDMI} = \mathrm{Index}(\mathrm{NIR}, \mathrm{SWIR}) = \frac{\mathrm{NIR} - \mathrm{SWIR}}{\mathrm{NIR} + \mathrm{SWIR}}
\]

W przypadku danych Sentinel-2, indeks ten wyznaczamy na podstawie kanałów B8 (NIR) i B11 (SWIR):
\[
\mathrm{NDMI} = \mathrm{Index}(\mathrm{B8}, \mathrm{B11}) = \frac{\mathrm{B8} - \mathrm{B11}}{\mathrm{B8} + \mathrm{B11}}
\]

\subsection{RED, NIR, SWIR czyli B4, B8, B11}

\textbf{RED} to czerwony kanał światła, silnie odbijany przez martwe liście. Wykorzystuje się go do identyfikacji typów roślinności, gleb oraz obszarów zabudowanych. Charakteryzuje się ograniczoną penetracją w wodzie i słabym odbiciem od liści zawierających chlorofil (żywe liście).

W satelitach Sentinel-2 kanał RED odpowiada pasmu B4: \cite{sentinel2_band_B4}
\begin{itemize}
    \item Rozdzielczość: 10 m/px
    \item Centralna długość fali: 665 nm
    \item Szerokość pasma: 30 nm
\end{itemize}

\textbf{NIR} (Near Infrared) to bliska podczerwień, która dobrze obrazuje linie brzegowe oraz zawartość biomasy.

W Sentinel-2 kanał NIR odpowiada pasmu B8: \cite{sentinel2_band_B8}
\begin{itemize}
    \item Rozdzielczość: 10 m/px
    \item Centralna długość fali: 842 nm
    \item Szerokość pasma: 115 nm
\end{itemize}

\textbf{SWIR} (Short-Wave Infrared) jest to fala wysokowrażliwa na zawartosć wody w obiektach. Dlatego jest dobrym wskaźnikiem wilgotności w 

W Sentinel-2 kanał SWIR odpowiada pasmu B11: \cite{sentinel2_band_B11}
\begin{itemize}
    \item Rozdzielczość: 20 m/px
    \item Centralna długość fali: 1610 nm
    \item Szerokość pasma: 130 nm
\end{itemize}

Istotną dla nas informacją na temat pasm dostarczanych przez Sentinel-2 jest rozdzielczość. Możemy zauważyć różnice w rozdzielczości między kanałami NIR i SWIR. W związku z tym musimy dokońać przeskalowania jednej z tych rozdzielczości do rozdzielczości drugiej.

\subsection{Przeskalowanie NIR do rozdzielczości SWIR}
Ponieważ kanały NIR i SWIR w Sentinel-2 mają różne rozdzielczości przestrzenne, przed obliczeniem wskaźnika NDMI konieczne jest dopasowanie ich do wspólnej siatki. Aby uzyskać jak najlepszą jakość wynikowego wskaźnika, zdecydowaliśmy się przeskalować kanał NIR (B8, 10 m/px) do rozdzielczości kanału SWIR (B11, 20 m/px). 
Wybraliśmy NIR ponieważ przy jego downscalingu uzyskamy dokładniejsze wyniki.
Do tego celu zastosowaliśmy metodę "area-based resampling" \cite{area_based_resampling}, która pozwala na bezpieczne zmniejszenie rozdzielczości obrazu przy zachowaniu reprezentatywności wartości pikseli.

Metoda ta polega na uśrednianiu wartości pikseli z obrazu o wyższej rozdzielczości, aby uzyskać odpowiadający im piksel w obrazie o niższej rozdzielczości. Matematycznie można to zapisać następująco:
\[
D_{i,j} = \frac{1}{4} \sum_{m=0}^{1} \sum_{n=0}^{1} I_{2i+m,\,2j+n}
\]
gdzie $D_{i,j}$ oznacza piksel w obrazie o niższej rozdzielczości, a $I_{2i+m,\,2j+n}$ to odpowiadające mu piksele w obrazie o wyższej rozdzielczości.

\subsection{Zrównoleglenie / rozproszenie obliczeń}

W związku z tym że chcielibyśmy by aplikacja działała na jednym komputrze wybraliśmy scierzkę zrównoleglenia obliczeń.
Następnym wyborem jest obliczenia wykonywane na CPU czy GPU.
Zadanie polega głównie na wykonywaniu prostych obliczeń arytmetycznych dla poszczególnych pikseli.
Operacje są powtarzalne oraz wykonywane niezależnie od siebie.
W związku z tym lepszą opcją będzie wykonanie obliczeń przy urzyciu GPU.

\subsection{Następnym krokiem będzie uzyskanie dostępu do danych z Sentinel-2}

Do tego celu skorzystaliśmy z API Sentinel-Hub. \cite{sentinel2_api_docs} 
Udostępnia on dane z satelity poprzez API dopiero po uwierzytelnieniu. 
W tym celu musielismy przejść przez process opisany tutaj \cite{sentinel2_api_docs_auth}.
Po jego przejsciu mamy dostep do Client ID oraz Client Secret które wykorzystamy do połączenia się z API.

API posiada reate limiting które są w miare restrykcyjne. \cite{sentinel2_api_auth_rate_limiting}
Z tego też powodu zamierzamy wykorzystać cache do przechowywania pobranych już danych.
Z natury projektu i dania sobie możliwości testowania wyników zamierzamy przechowywać nieprzetworzone dane zamiast obliczonych już wyników.
Pomoże nam to przeprowadzić testy i wprowadzić poprawki w przyszłości.

\subsection{Wybór technologii}

Jako że określiliśmy już dokładniej ramy projektu możemy potwierdzić wybrane przez nas technologie. 
Ogólny język implementacji to Python.
Do pobierania danych z api wykorzystaliśmy bibliotekę SentinelHub. Która udosptępnia już gotowe funkcje pozwalające zautoryzować się z api oraz pobrać dane.
Do gui wykorzystaliśmy Tkinter podstawową i dosyć prostą bibliotekę do tworzenia interfejsu graficznego.
Do prezentacji mapy wykorzystaliśmy TkinterMapView.
Do zrównoleglenia obliczeń skorzystamy z Taichi Lang \cite{taichi_lang_docs} którego zaembedujemy w pythonie.
Jest to język programowania osadzony w pythonnie stworzony do wysokowydajnych obliczeń.
Jego zaletą jest to że kod napisany raz przekąpiluje się na wiele różnych platform.



\section{Przebieg procesu w aplikacji}
\section{Równoległe przetwarzanie}
\subsection{Które obliczenia i dlaczego postanowilismy zrownoleglic / rozproszyc}
\subsection{Jak dokonaliśmy zrownoleglenia obliczeń}
\subsection{Jakie wyniki uzyskaliśmy}
% miejsce na wykresy i statystyki
% miejsce na testy i benchmarki

\subsection{Kiedy przyspieszenie ma sens, a kiedy nie ma}
\subsection{Jakie wnioski można wyciągnąć z uzyskanych wyników}

\section{Interpretacja i wnioski}

\printbibliography[heading=bibintoc]

\end{document}